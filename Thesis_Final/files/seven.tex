%% Copyright 1998 Pepe Kubon
%%
%% `two.tex' --- 2nd chapter for thes-full.tex, thes-short-tex from
%%               the `csthesis' bundle
%%
%% You are allowed to distribute this file together with all files
%% mentioned in READ.ME.
%%
%% You are not allowed to modify its contents.
%%

%%%%%%%%%%%%%%%%%%%%%%%%%%%%%%%%%%%%%%%%%%%%%%%%%
%
%     Chapter 7  
%
%%%%%%%%%%%%%%%%%%%%%%%%%%%%%%%%%%%%%%%%%%%%%%%%

\chapter{Conclusions}
\label{ch:con}
%\section{Summary of the Thesis}

The skyline subspace problem is originally motivated by the problem of what distinguishes one from its peers in social networks~\cite{lo2013distinguish}. In this thesis, we formulate the social network into a graph with labels and consider the distances between a person and the labels as the factors that distinguish the person from its peers. We propose a bottom-up algorithm to answer \emph{skyline subspace query} which is based on set enumeration and dominating candidate sets intersection. To tackle the problems of skyline subspaces on graph and the skyline subspaces on Euclidean space, we develop effective pruning methods to reduce the search space. 
We do empirical studies using both synthetic and real datasets to evaluate our approach. We generate the synthetic graph based on the Kronecker graph model and the real world datasets are from DBLP and YELP. The experimental results verify the efficiency of our algorithms.

As for future work, we can consider the following directions.  

\begin{itemize}
   
\item \textit{Using top-down set enumeration.} Our algorithm is based on bottom-up set enumeration. In bottom-up manner, we take the advantage of the property that if a target skyline subspace is found then we do not need to search for the subspaces that contain this subspace. For top-down approach, one of the advantage we can take is that if the query point is strictly dominated by some points in a certain subspace $\mathcal{A}$, then we do not need to check the subsets of the subspace $\mathcal{A}$ because we know that the query point cannot be a skyline point in those subspaces.

\item \textit{Further pruning method development in Euclidean space.} There are many properties in euclidean space. In~\cite{sharifzadeh2006spatial}, Sharifzadeh et al. took the advantage of the property of convex hull to reduce the size of skyline candidates. They also used the Voronoi diagram structure to index the graph. For future work, we can index the spatial points using Voronoi diagram instead of R-tree and apply pruning methods based on some geometry properties such as the property of convex hull.

\item \textit{Skyline subspace algorithm on other applications.} Finding the skyline subspaces on road networks and metric space is still an open problem. This topic is an interesting problem to be studied in the future.

\end{itemize}














