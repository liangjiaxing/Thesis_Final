%% Copyright 1998 Pepe Kubon
%%
%% `two.tex' --- 2nd chapter for thes-full.tex, thes-short-tex from
%%               the `csthesis' bundle
%%
%% You are allowed to distribute this file together with all files
%% mentioned in READ.ME.
%%
%% You are not allowed to modify its contents.
%%

%%%%%%%%%%%%%%%%%%%%%%%%%%%%%%%%%%%%%%%%%%%%%%%%%
%
%     Chapter 5   
%
%%%%%%%%%%%%%%%%%%%%%%%%%%%%%%%%%%%%%%%%%%%%%%%%

\chapter{Spatial Skyline Subspace}
\label{ch:spatial_skyline_subspace}

In this chapter, we will introduce the algorithms to compute the spatial skyline subspace queries. In the computation of spatial skyline subspace skyline queries, we use the same set enumeration framework as shown in Chapter~\ref{ch:graph}. However, in this chapter we introduce a different method to prune the unnecessary \emph{dominating candidates} based on some geometric property of skyline subspace.

\section{Label Collecting in Radius $D$}

We assume that the data points are indexed in R-tree. In R-tree we can get all points in a certain rectangle efficiently. By indexing the data points in R-tree, if we want access the neighbouring points of certain query point $q$, we can query the square with center $q$ in R-tree without accessing the whole set of data points. We compute query vertex $q$'s label distance vector $LV_q$ by checking the labels of the points in rectangle with upper-left corner point $(p.x - D, p.y - D)$ and lower-right corner point $(p.x + D, p.y + D)$. Then we collect the labels within distance $D$ as the label distance vector $LV_q$.

\section{Dominating Candidates in Spatial Subspace Skyline}

\begin{algorithm}[H]
  \caption{Dominating Candidates}\label{algo:blah}
  \begin{algorithmic}[1]
  \show\LOOP
    \REQUIRE R-tree $R$ that indexes the spatial points and label distance vector $LV_q$ of query vertex $q$.
    \ENSURE Dominating Candidates $CAND$ of $1$-dimension subspace;
    \FORALL {$\left(l, dist\right)$ in $LV_q$}
        \FORALL {point $p$ contains label $l$}
            \STATE look at the points in rectangle $rec = (p.x-dist, p.y-dist, p.x+dist, p.y+dist)$
            \STATE push all the points in $rec$ in distance $dist$ to $CAND_l$
        \ENDFOR
    \ENDFOR
  \end{algorithmic}
\end{algorithm}

In this algorithm, we 




