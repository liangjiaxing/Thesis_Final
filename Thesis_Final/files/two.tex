%% Copyright 1998 Pepe Kubon
%%
%% `two.tex' --- 2nd chapter for thes-full.tex, thes-short-tex from
%%               the `csthesis' bundle
%%
%% You are allowed to distribute this file together with all files
%% mentioned in READ.ME.
%%Discuss the reality where queries are not randomly generated.
%% You are not allowed to modify its contents.
%%

%%%%%%%%%%%%%%%%%%%%%%%%%%%%%%%%%%%%%%%%%%%%%%%%%
%
%     Chapter 2   
%
%%%%%%%%%%%%%%%%%%%%%%%%%%%%%%%%%%%%%%%%%%%%%%%%

\chapter{Related Work}
\label{ch:related-work}

Our problem of \emph{skyline subspace query} is mainly related to the existing work on general skyline query, subspace skyline computation and skyline query with specific constrain which are reviewed in Section~\ref{sec:rel:general}, Section~\ref{sec:rel:subspace} and Section~\ref{sec:rel:constrain}, respectively.

\section{General Skyline Query}
\label{sec:rel:general}

There is a number of studies of skyline in Data Mining area. The problem of finding the maxima (skyline) of a set of vectors is first investigated in~\cite{kung1975finding} where an $O(n\log ^{d-2}n)$ algorithm for dimension $d\geq 4$ and an $O(n\log n)$ time algorithm for dimension $d = 2, 3$ are proposed. To integrate the skyline operator into database, Borzsony et al.~\cite{borzsony2001skyline} proposed the Block-nested-loops Algorithm (BNL) and Divide and Conquer Algorithm (DC) to compute the skyline queries. BNL essentially maintain a window of incomparable and compare each object in the database with the objects in the window and outputs the skyline at the end of the iteration. DC divides the dataset into several partitions and each partition can fit in memory. The skylines in all partitions are computed individually in main memory, and then merged to produce the final skyline objects. 
The sort-first-skyline (SFS)~\cite{chomicki2003skyline} algorithm also maintains a window of object which is similar to the BNL algorithm. In addition to that, it sorts the input data first so that it can guarantee the objects in the window can be output immediately as skyline points which makes the algorithm more efficient.
Different from original skyline query problem which need to scan the whole database to output all the skyline points at the very end, progressive skyline problem is to progressively return the skyline points as they are identified. Tan et al.~\cite{tan2001efficient} proposed the Bitmap algorithm and Index algorithm to tackle this problem. The Bitmap algorithm is completely exploits a bitmap structure to identify whether a point is an skyline point. Index algorithm transfer the multi-dimensional objects into $1$-dimensional space and store the objects in a B+-tree structure. To explore the progressive skyline problem further, Tao et al.~\cite{papadias2005progressive} develop bound-and-branch skyline (BBS) which takes the advantage of R-tree to search for the nearest neighborhood. These works are about finding the skyline points efficiently in Database.

\section{Subspace skyline computation}
\label{sec:rel:subspace}

For subspace skyline problem, Pei et al.~\cite{pei2005catching} proposed the \emph{Skyey} algorithm based on the property of decisive subspaces to compute the skyline points for every subspace. Besides outputting the skyline points, the algorithm return the skyline groups and their signatures. \emph{Skyey} algorithm also takes the advantage of sharing sorted order of the objects among different subspaces. Yuan et al.~\cite{yuan2005efficient} developed the \emph{Top-Down Skyline Algorithm} to compute the skyline in every subspace. They also developed a novel data structure \emph{skylist} to store the skyline objects in different subspaces in a compact way. 
Tao et al~\cite{tao2006subsky} proposed a technique SUBSKY to tackle the subspace skyline problem using a single B-tree. They apply an transformation on the multi-dimensional data to $1$-dimensional value to enable several effective pruning heuristics.
The subspace skyline queries on high dimensional data was studied in~\cite{jin2007efficient}. To answer subspace skyline queries on high dimensional data, Jin et al~\cite{jin2007efficient} proposed novel notions of \emph{maximal partial-dominating space}, \emph{maximal partial-dominated space and the maximal equality space} between pairs of skyline objects in the full space. Both of their algorithms are in the top-down manner. In our paper, we are focusing on skyline subspace of one query point but not the whole dataset. Our work is to find the subspaces in terms of the query point. We also extend our work on the graph setting and spatial setting.

\section{Skyline Query with Specific Constrain}
\label{sec:rel:constrain}

Sharifzadeh et al.~\cite{sharifzadeh2006spatial} proposed an efficient algorithm $VS^2$ to compute the spatial skyline based on some geometry properties of convex hull and voronoi diagram. They proved that any point whose Voronoi cell intersects with boundaries of convex hull of the query points is a skyline point.

Zou et al.~\cite{zou2010dynamic} applied the skyline algorithm on computing the skyline queries in the graph structure. They proposed the SSP pruning method to make queries processing more efficient.

Both of their works are focusing on finding the skylines while our work is focusing on finding the subspaces.




