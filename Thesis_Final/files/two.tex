%% Copyright 1998 Pepe Kubon
%%
%% `two.tex' --- 2nd chapter for thes-full.tex, thes-short-tex from
%%               the `csthesis' bundle
%%
%% You are allowed to distribute this file together with all files
%% mentioned in READ.ME.
%%Discuss the reality where queries are not randomly generated.
%% You are not allowed to modify its contents.
%%

%%%%%%%%%%%%%%%%%%%%%%%%%%%%%%%%%%%%%%%%%%%%%%%%%
%
%     Chapter 2   
%
%%%%%%%%%%%%%%%%%%%%%%%%%%%%%%%%%%%%%%%%%%%%%%%%

\chapter{Related Work}
\label{ch:related-work}

Our problem of \emph{skyline subspace query} is mainly related to the existing work on general skyline query, subspace skyline computation and skyline query with specific constrain which are reviewed in Section~\ref{sec:rel:general}, Section~\ref{sec:rel:subspace} and Section~\ref{sec:rel:constrain}, respectively.

\section{General Skyline Query}
\label{sec:rel:general}
The general skyline problem is to find all the points that are not dominated by any other points.
There is a number of studies of skyline in Data Mining area. The problem of finding the maxima (skyline) of a set of vectors is first investigated in~\cite{kung1975finding} where an $O(n\log ^{d-2}n)$ algorithm for dimension $d\geq 4$ and an $O(n\log n)$ time algorithm for dimension $d = 2, 3$ are proposed. The algorithm in~\cite{kung1975finding} is based on the divide and conquer principle. To integrate the skyline operator into database, Borzsony et al.~\cite{borzsony2001skyline} proposed the Block-nested-loops Algorithm (BNL) and Divide and Conquer Algorithm (DC) to compute the skyline queries. BNL essentially maintain a window of incomparable and compare each object in the database with the objects in the window and outputs the skyline at the end of the iteration. DC divides the dataset into several partitions and each partition can fit in memory. The skylines in all partitions are computed individually in main memory, and then merged to produce the final skyline objects. 
The sort-first-skyline (SFS)~\cite{chomicki2003skyline} algorithm also maintains a window of object which is similar to the BNL algorithm. In addition to that, it sorts the input data first so that it can guarantee the objects in the window can be output immediately as skyline points which makes the algorithm more efficient.
Kossmann et al.~\cite{kossmann2002shooting} studied the relationship between nearest neighbour and skyline points and developed the NN algorithm to compute skyline using R-tree~\cite{beckmann1990r} to index the data points.
Different from original skyline query problem which need to scan the whole database to output all the skyline points at the very end, progressive skyline problem is to progressively return the skyline points as they are identified. Tan et al.~\cite{tan2001efficient} proposed the Bitmap algorithm and Index algorithm to tackle this problem. The Bitmap algorithm exploits a bitmap structure to identify whether a point is a skyline point. The Index algorithm transfer the multi-dimensional objects into $1$-dimensional space and store the objects in a B+-tree structure. To explore the progressive skyline problem further, Papadias et al.~\cite{papadias2003optimal, papadias2005progressive} develops the bound-and-branch skyline (BBS) algorithm which takes the advantage of R-tree~\cite{beckmann1990r} to search for the nearest neighborhood. These works are about finding the skyline points efficiently in Database. 

\section{Subspace Skyline Computation}
\label{sec:rel:subspace}
The research of subspace skyline problem is to study the relationship between the property of skyline and subspace. The major research problem of this field is that given a set of $n$-dimensional points, we want to compute all the skyline points in all the subspaces of the full space.

For the subspace skyline problem, Pei et al.~\cite{pei2005catching} proposed the \emph{Skyey} algorithm based on the property of decisive subspaces to compute the skyline points for every subspace.  The algorithm not only outputs the skyline points in every subspace, but it also returns the skyline groups and their signatures.
Getting the information of the skyline group and signatures of a skyline point could show the combinations of factors on which an object dominates the other objects. \emph{Skyey} algorithm takes the advantage of sharing sorted order of the objects among different subspaces to make the computation efficient. Yuan et al.~\cite{yuan2005efficient} developed the \emph{Top-Down Skyline Algorithm} to compute the skyline in every subspace. They also developed a novel data structure \emph{skylist} to store the skyline objects in different subspaces in a compact way. 
Both of their algorithms are in the top-down manner.

Most of time, people are interested in computing the skyline of one particular subspace instead of all subspaces. To tackle the problem of computing skyline in one particular subspace, Tao et al.~\cite{tao2006subsky} proposed the SUBSKY algorithm using a single B-tree. They apply a transformation on the multi-dimensional data to $1$-dimensional value to enable several effective pruning heuristics.
The subspace skyline queries on high dimensional data was studied in~\cite{jin2007efficient}. To study the subspace skyline queries on high dimensional data, Jin et al.~\cite{jin2007efficient} proposed novel notions of \emph{maximal partial-dominating space}, \emph{maximal partial-dominated space and the maximal equality space} between pairs of skyline objects in the full space. In our paper, we are focusing on skyline subspace of one query point but not the whole dataset. Different from previous works of subspace skyline. Our work is to find the subspaces in terms of the query point. We also extend our work on the graph setting and spatial setting.

\section{Skyline Query in Specific Scenarios}
\label{sec:rel:constrain}
In regular skyline problem, the $n$-dimensional values of the data points is known. However, in many scenarios, the values of the data points is unknown at the beginning and somehow it is costly to compute the actual values of all the data points. Finding efficient ways to compute the skyline points in those scenarios are also interesting problems to study. In this section, we will introduce several papers that study the skyline query problem in different scenarios.

To explore the relationship Euclidean geometry and skyline, the spatial skyline problem is studied in~\cite{sharifzadeh2006spatial}: Given the two sets $P$ of data points and $Q$ of query points, the spatial skyline of $P$ with respect to $Q$ is the set of those points in $P$ whose distances to every point in $Q$ are not dominated by any other point of $P$. Sharifzadeh et al.~\cite{sharifzadeh2006spatial} proved two important theorems of spatial skyline problem: Any point $p \in P$ which is inside the convex hull of $Q$ is a skyline point. Any point whose Voronoi cell intersects with boundaries of convex hull of the query points is a skyline point. They proposed the algorithm $B^2S^2$ and $VS^2$ to compute the spatial skyline based on the geometry properties of convex hull and Voronoi diagram. $B^2S^2$ algorithm is based on Branch and Bound algorithm which stores the points in R-tree. $VS^2$ stores the points in Voronoi diagram and takes the advantage of Voronoi diagram to compute the nearest neighbours efficiently.

Road network skyline problem is studied in~\cite{deng2007multi}: Given a road network modeled as a graph $G=(E, V)$ and a set of query points $Q$, the road network skyline query of $V$ with respect to $Q$ is the set of those points in $V$ whose distances to every point in $Q$ are not dominated by any other point of $V$. Deng et al.~\cite{deng2007multi} proposed Collaborative Expansion Algorithm (CE), Euclidean Distance Constraint Algorithm (EDC) and Lower-Bound Constraint Algorithm (LBC) to solve the problem. CE is a straight forward method which is based on Dijkstra Algorithm without taking the geometry information of the vertices into account. EDC algorithm is based on A* algorithm which takes the advantage of the property that the spatial distance between two vertices is less than the road distance between them. LBC algorithm decreases the skyline point candidate size by introducing the concept of \emph{path distance lower bound}. Both EDC and LBC algorithm utilize Euclidean distance as the lower bound of shortest path distance in road network to perform pruning. 

Both spatial skyline problem and road network skyline problem are the particular types of metric skyline problem. Metric distance satisfies the triangle inequality: $dist(x, z) \leq dist(x, y) + dist(y, z)$. Metric distance is the general case of spatial Euclidean distance and road network distance. Chen et al. illustrate a triangle-based pruning mechanism to answer metric skyline queries through a metric index with M-tree in~\cite{chen2008dynamic}. To go further on the problem of skyline computation on metric space, Fuhry et al.~\cite{fuhry2009efficient} proved that the distance between a metric space skyline point and $q$ is bounded by $2r_q + d(q, NN(q))$ where $r_q$ is the radius of the enclosing ball for the set of query point $Q$ in terms of the query point $q$. Using this property, they proposed $N^2RS$ algorithm which only search for the skyline points in a certain range. $B^2MS^2$ algorithm based on a generic index tree is also proposed in~\cite{fuhry2009efficient}.

In road network skyline problem, Deng et al.~\cite{deng2007multi} takes the advantage of the geometry spatial information. To solve the skyline problem on pure graph without any spatial information, Zou et al.~\cite{zou2010dynamic} proposed the SSP Query algorithm to tackle this problem. They first introduced the concept of \emph{1−Hop Shortest Path Tree} to index the tree in the database in order to compute the shortest-distance query efficiently. They also introduced the SSP pruning method to prune the unnecessary skyline point candidates. After getting the skyline point candidates, they perform the BNL algorithm~\cite{borzsony2001skyline} to find the skyline points.

\section{Reverse Dynamic Skyline Query}
All of the previous works are focusing on finding the skyline points themselves. Papadias et al.~\cite{papadias2003optimal} introduced the concept of \emph{dynamic skyline query}: Given a query point $q$ and a data set $P$, a Dynamic Skyline Query according to $q$ returns all data points in $P$ that are not dynamically dominated. A point $p_1 \in P$ dynamically dominates $p_2 in P$ with regard to the query point $q$ if for all $i: |q^i-p^i_1| \leq |q^i-p^i_2|$ and at least one $j: |q^j-p^j_1| < |q^j-p^j_2|$. Dellis et al.~\cite{dellis2007efficient} introduced an opposite version of this problem \emph{reverse skyline query}: Given a query point $q$ and a data set $P$, a Reverse skyline query according to $q$ returns all data points $p_1 \in P$ where $q$ is in the dynamic skyline of $p_1$. They proposed BBRS algorithm, which is an improved customization of the original BBS~\cite{papadias2003optimal} algorithm to tackle this problem.

In our paper, we study the opposite version of the problem in~\cite{tao2006subsky}: Given a set of data points and a query point, we want to compute all the subspaces where the query point is not dominated by any other point. We also extend our work on finding the skyline subspace on the graph and Euclidean space.



